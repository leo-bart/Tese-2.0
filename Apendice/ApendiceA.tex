\chapter{Dedução da matriz de massa na formulação em coordenadas
flutuantes \label{ch:apendice A}}

Na FCF a posição de um ponto qualquer em um corpo flexível é dada por
\begin{equation}
 \vec{r} = \vec{r}_O +  \tens{A}(\vec{u}^* + \vec{w}^*) \label{eq-ApA: posição}
\end{equation}
em que o sobrescrito $^*$ indica que o vetor está representado em coordenadas
locais, enquanto sua ausência denota coordenadas globais. Derivando a relação
\eqref{eq-ApA: posição}, encontra-se para a velocidade:
\begin{equation}
 \dot{\vec{r}} = \dot{\vec{r}}_O +  \dot{\tens{A}}(\vec{u}^* + \vec{w}^*) +
\tens{A}\vec{w}^* \label{eq-ApA: velocidade}
\end{equation}

A energia cinética é dada pela integral
\begin{equation}
 \mathcal{T} = \frac{1}{2} \int{\rho \tp{\dot{\vec{r}}} \dot{\vec{r}} \, dV}
\end{equation}
que envolve a norma ao quadrado do vetor $\dot{\vec{r}}$. Lembrando que
$\dot{\tens{A}}\tp{\tens{A}} = \tilde{\vec{\omega}}$, essa última quantidade
pode ser calculada como
\begin{align}
 \tp{\dot{\vec{r}}}\dot{\vec{r}} =&  \tp{\dot{\vec{r}}}_O\dot{\vec{r}}_O
+ \tp{\dot{\vec{r}}}_O \tilde{\vec{\omega}}(\vec{u} + \vec{w}) +
\tp{\dot{\vec{r}}}_O \dot{\vec{w}} - \\
&(\tp{\vec{u}} + \tp{\vec{w}})\tilde{\vec{\omega}}\dot{\vec{r}}_O -
(\tp{\vec{u}}
+ \tp{\vec{w}})\tilde{\vec{\omega}}\tilde{\vec{\omega}}(\vec{u} + \vec{w})
- (\tp{\vec{u}} + \tp{\vec{w}})\tilde{\vec{\omega}}\dot{\vec{w}} + \\
&\tp{\dot{\vec{w}}}\dot{\vec{r}}_O +
\tp{\dot{\vec{w}}}\tilde{\vec{\omega}}(\vec{u}+\vec{w}) +
\tp{\dot{\vec{w}}}\dot{\vec{w}} \label{eq:apA - rr passo 1}
\end{align}

Na FCF, o campo de deslocamentos é localmente expresso como $\vec{w}
= \vec{S}\vec{e}$, com  $\vec{S}$ sendo a matriz com funções de forma e
$\vec{e}$, as coordenadas nodais em relação ao sistema local de coordenadas.
Define-se um vetor de coordenadas generalizadas tal que:
\begin{equation}
 \vec{q}_f = \tp{\begin{bmatrix}
                      \tp{\vec{r}}_O & \tp{\vec{\theta}} & \tp{e}
                     \end{bmatrix}}
\end{equation}
em que $\vec{\theta}$ são os parâmetros de rotação do sistema local em relação
ao global. Independentemente de quais esses parâmetros, existe uma relação
linear entre suas derivadas e a velocidade angular $\vec{\omega}$:
\begin{equation}
 \vec{\omega} = \vec{G}\dot{\vec{\theta}} \label{eq-ApA:theta-pt para omega}
\end{equation}

Para facilitar a montagem da matriz de massa, definem-se as matrizes simétricas
auxiliares $\vec{N}_{11}$, $\vec{N}_{22}$ e $\vec{N}_{33}$:
\begin{align}
 \vec{N}_{11} = \begin{bmatrix}
            \vec{E} & \vec{0} & \vec{0} \\
            \vec{0} & \vec{0} & \vec{0}\\
            \vec{0} & \vec{0} & \vec{0}
           \end{bmatrix},
 \vec{N}_{22} = \begin{bmatrix}
            \vec{0} & \vec{0} & \vec{0} \\
            \vec{0} & \vec{E} & \vec{0}\\
            \vec{0} & \vec{0} & \vec{0}
           \end{bmatrix},
 \vec{N}_{33} = \begin{bmatrix}
            \vec{0} & \vec{0} & \vec{0} \\
            \vec{0} & \vec{0} & \vec{0}\\
            \vec{0} & \vec{0} & \vec{E}
           \end{bmatrix}
\end{align}

Utilizando a relação \eqref{eq-ApA:theta-pt para omega} e $\vec{N}_{22}$ vem
que
\begin{subequations}
\label{eq-apA:relações entre velocidades generalizadas}
\begin{align}
 \dot{\vec{r}} &= \vec{N}_{11} \dot{\vec{q}}_f \label{eq-apA:r = Nq}\\
 \dot{\vec{\theta}} &= \vec{G}\vec{N}_{22} \dot{\vec{q}}_f  \label{eq-apA:omega
= Nq}\\
 \dot{\vec{e}} &= \vec{N}_{33} \dot{\vec{q}}_f \label{eq-apA:e = Nq}
\end{align}
\end{subequations}

A partir das Eqs.~\eqref{eq-apA:r = Nq}, \eqref{eq-apA:omega = Nq} e
\eqref{eq-apA:e = Nq}, a multiplicação de velocidades apresentada em
\eqref{eq:apA - rr passo 1} pode ser reformulada:
\begin{align}
 \tp{\dot{\vec{r}}}\dot{\vec{r}} =&
\tp{\dot{\vec{q}}}_f \vec{N}_{11} \dot{\vec{q}}_f + \\&
+\tp{\dot{\vec{q}}}_f \vec{N}_{11} \tilde{\vec{\omega}} (\vec{u} +
\vec{S}\vec{N}_{33}\vec{q}_f) + \\&
+\tp{\dot{\vec{q}}}_f \vec{N}_{11}\vec{S}\vec{N}_{33}\dot{\vec{q}}_f + \\&
-(\tp{\vec{u}} + \tp{\vec{q}}_f\vec{N}_{33}\tp{\vec{S}})
\tilde{\vec{\omega}} \vec{N}_{11} \dot{\vec{q}}_f + \\&
-(\tp{\vec{u}} + \tp{\vec{q}}_f\vec{N}_{33}\tp{\vec{S}})
\tilde{\vec{\omega}}\tilde{\vec{\omega}}(\vec{u} +
\vec{S}\vec{N}_{33}\vec{q}_f) +\\&
-(\tp{\vec{u}} + \tp{\vec{q}}_f\vec{N}_{33}\tp{\vec{S}})
\tilde{\vec{\omega}}\vec{S}\vec{N}_{33}\dot{\vec{q}}_f + \\&
+
\tp{\dot{\vec{q}}}_f\vec{N}_{33}\tp{\vec{S}}\vec{N}_{11}\dot{\vec{q}}_f
+ \\&
+
\tp{\dot{\vec{q}}}_f\vec{N}_{33}\tp{\vec{S}}\tilde{\vec{\omega}}(\vec{u
} + \vec{S}\vec{N}_{33}\vec{q}_f) + \\&
+\tp{\dot{\vec{q}}}_f\vec{N}_{33}\tp{\vec{S}}\vec{S}\vec{N}_{33}\dot{
\vec{q}}_f
\end{align}

Substutuindo a expressão acima na equação da energia cinética e utilizando a
propriedade do produto vetorial $\tilde{\vec{\omega}}\vec{v} = -
\tilde{\vec{v}}\vec{\omega}$ chega-se a:
\begin{align}
 \mathcal{T} = \frac{1}{2} \tp{\dot{\vec{q}}_f} {\vec{M}^*}
\dot{\vec{q}}_f = \frac{1}{2} \tp{\dot{\vec{q}}_f} \begin{bmatrix}
           \vec{M}^*_{11} & \vec{M}^*_{12} & \vec{M}^*_{13} \\
	   \vec{M}^*_{12} & \vec{M}^*_{22} & \vec{M}^*_{23} \\
           \vec{M}^*_{13} & \vec{M}^*_{23} & \vec{M}^*_{33}
           \end{bmatrix}
\dot{\vec{q}}_f
\end{align}
em que as matrizes $\vec{M}^*_{ij}$ são funções das coordenadas
generalizadas $\vec{q}_f$ dadas por:
\begin{align}
 \vec{M}^*_{11} &= m\vec{E} \\
 \vec{M}^*_{12} &= - \int{\rho \, \mathcal{A}({\vec{u} +
\vec{S}\vec{N}_{33}\vec{q}_f})\vec{G}dV} \\
 \vec{M}^*_{13} &= \int{\rho \vec{S} dV} \\
 \vec{M}^*_{22} &= - \int{\rho \tp{\vec{G}}
\,\mathcal{A}(\tp{\vec{u}} + \tp{\vec{q}}_f \vec{N}_{33}
\tp{\vec{S}}) \mathcal{A}(\vec{u} + \vec{S}\vec{N}_{33}\vec{q}_f)\vec{G}
dV} \\
 \vec{M}^*_{23} &= - \int{\rho \, \mathcal{A}(\vec{u} +
\vec{S}\vec{N}_{33}\vec{q}_f)\vec{G}dV} \\
 \vec{M}^*_{33} &= \int{\rho \tp{\vec{S}}\vec{S}dv}
\end{align}
Nas equações acima, o operador $\mathcal{A}(\vec{x}) = \tilde{\vec{x}}$ é a representação antissimétrica do vetor $\vec{x}$.

Note-se que ${\vec{M}^*}$ depende apenas da posição, mas não da
velocidade. Por isso, mesmo não sendo constante, tal matriz ainda pode ser entendida como uma matriz de massa, uma
vez que multiplicará as acelerações após o cálculo do valor estacionário da
função Lagrangeana.








