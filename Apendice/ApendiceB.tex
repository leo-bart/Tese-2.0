 \chapter{Dedução das forças nodais a partir do princípio dos trabalhos virtuais}
No que se segue uma dedução alternativa do vetor de forças nodais nos elementos de viga
em coordenadas nodais absolutas é apresentada. Aqui utiliza-se o princípio das potências
virtuais como ponto de partida, ao invés da energia de deformação (abordagem que foi utilizada
no corpo do texto).

Por meio de uma hipótese de pequenos deslocamentos, é possível chegar a uma matriz de rigidez
constante.

Aplicando-se o teorema da divergência na Eq.~\eqref{eq: jourdain_ANCF}, chega-se a:
\begin{equation}
    \delta'\dot{\vec{q}}^T \vec{M}_i \ddot{\vec{q}} + 
    \int_{e^{(i)}}{\vec{\sigma}:\left(\nabla{\delta'\dot{\vec{r}}}\right) dV} 
    - \int_{\Gamma^{(i)}}{\vec{\sigma}\vec{n} \cdot \delta'\dot{\vec{r}} d\Gamma} - \int_{e^{(i)}}{\vec{b}\cdot \delta'\dot{\vec{r}} dV} = 0
\end{equation}
em que $\Gamma^{(i)}$ indica as regiões da superfície do elemento com vetor normal $\vec{n}$ e que estão 
sujeitas a condições de contorno de 
Neumann, ou seja, a tensões externas tais que $\vec{t}=\vec{\sigma}\vec{n}$. Obtém-se, assim:
\begin{equation}
    \delta'\dot{\vec{q}}^T \vec{M}_i \ddot{\vec{q}} + 
    \int_{e^{(i)}}{\vec{\sigma}:\left(\nabla{\delta'\dot{\vec{r}}}\right) dV} 
    - \int_{\Gamma^{(i)}}{\vec{t} \cdot \delta'\dot{\vec{r}} d\Gamma} - \int_{e^{(i)}}{\vec{b}\cdot \delta'\dot{\vec{r}} dV} = 0
\end{equation}

Admitindo que o material do elemento é elástico linear, pode-se seguir a dedução de \citeonline[p. 505]{bittencourt_computational_2015}
para encontrar:
\begin{equation}
\begin{split}
     \delta'\dot{\vec{q}}^T \vec{M}_i \ddot{\vec{q}} + 
    \int_{e^{(i)}}{\left\{ \frac{\mu}{2} \left[ 
        \left( \nabla{\vec{u}} + \nabla{\vec{u}}^T \right) : \left( \nabla{\delta'\dot{\vec{r}}} + \nabla{\delta'\dot{\vec{r}}}^T \right)
        + \lambda \left(\nabla{}\cdot\vec{u}\right) \left(\nabla{}\cdot\delta'\dot{\vec{r}}\right)
    \right] \right\} dV} +\\
    - \int_{\Gamma^{(i)}}{\vec{t} \cdot \delta'\dot{\vec{r}} d\Gamma} 
    - \int_{e^{(i)}}{\vec{b}\cdot \delta'\dot{\vec{r}} dV} = 0
\end{split} \label{eq: eq_mov_1}
\end{equation}
onde $\vec{u}$ é o campo de deslocamentos e $\mu$ e $\lambda$ são os parâmetros de Lamè do material.

Supondo-se pequenas deformações, pode-se obter uma representação matricial da Eq.~\eqref{eq: eq_mov_1} por meio
do tensor de pequenas deformações de Lagrange escrito em função do operador diferencial
$\vec{L}$ definido por
\begin{equation}
    \vec{L} = \begin{bmatrix}
        \frac{\partial }{\partial r_{0x}} & 0 & 0 \\
        0 & \frac{\partial }{\partial r_{0y}} & 0 \\
        0 & 0 & \frac{\partial }{\partial r_{0z}} \\
        \frac{\partial }{\partial r_{0y}} & \frac{\partial }{\partial r_{0x}} & 0 \\
        \frac{\partial }{\partial r_{0z}} & 0 & \frac{\partial }{\partial r_{0x}} \\
        0 & \frac{\partial }{\partial r_{0z}} & \frac{\partial }{\partial r_{0y}}
    \end{bmatrix} \label{eq: operador_L}
\end{equation}
e da matriz constitutiva $\vec{D}$:
\begin{equation}
    \vec{D} = \frac{E}{(1+\nu)(1-2\nu)}\begin{bmatrix}
        1-\nu & \nu & \nu & 0 & 0 & 0 \\
        \nu & 1-\nu & \nu & 0 & 0 & 0 \\
        \nu & \nu & 1-\nu & 0 & 0 & 0 \\
        0 & 0 & 0 & \frac{1-2\nu}{2} & 0 & 0 \\
        0 & 0 & 0 & 0 & \frac{1-2\nu}{2} & 0 \\
        0 & 0 & 0 & 0 & 0 &\frac{1-2\nu}{2}  \\
    \end{bmatrix} \label{eq: matriz_D}
\end{equation}
em que $E$ é o módulo de elasticidade e $\nu$ o coeficiente de Poisson do material. Pela substituição das relações
\eqref{eq: operador_L} e \eqref{eq: matriz_D} na equação de movimento \eqref{eq: eq_mov_1} obtém-se:
\begin{equation}
\begin{split}
     \delta'\dot{\vec{q}}^T \vec{M}_i \ddot{\vec{q}} + 
    \int_{e^{(i)}}{\delta'\dot{\vec{r}}^T \vec{L}^T \left(\vec{D}\vec{L}\vec{u}\right)  dV} +\\
    - \int_{\Gamma^{(i)}}{\vec{t} \cdot \delta'\dot{\vec{r}} d\Gamma} 
    - \int_{e^{(i)}}{\vec{b}\cdot \delta'\dot{\vec{r}} dV} = 0
\end{split} \label{eq: eq_mov_2}
\end{equation}

\begin{figure}[h]
 \centering
 \begin{tikzpicture}[use Hobby shortcut,scale=1.25]
    % sistema de coordenadas
    \draw[-stealth] (0,0) -- (1,0);
    \draw[-stealth] (0,0) -- (0,1);
    \draw[-stealth] (0,0) -- (-0.86,-0.5);


    % batatoide
    
    \begin{scope}[shift={(2,2)}]
        \path
          (0,0) coordinate (z0)
          (1.2,0) coordinate (z1)
          (1.5,1) coordinate (z2)
          (-0.2,1) coordinate (z3)
          (0.1,0.5) coordinate (z4);
        \coordinate (P) at (0.4,0.0);
        \coordinate (Q) at (0.55,1.6);
        \node[left]() at (P) {$P$};
        \node[left]() at (Q) {$Q$};
        \draw[thick,{Circle[sep=-2pt]}-stealth] (P) -- (Q) node[midway,right]{$\vec{d}$};
        \draw[closed,ultra thick] (z0) .. (z1) .. (z2) .. (z3) .. (z4);
    \end{scope}

    \begin{scope}[shift={(5,1)}, rotate=-20,xscale=1.2,yscale=1.32]
        \path
          (0,0) coordinate (z0)
          (1,0) coordinate (z1)
          (1.4,1) coordinate (z2)
          (-0.2,1) coordinate (z3)
          (0.1,0.5) coordinate (z4);
        \coordinate (P') at (0.4,0.0);
        \coordinate (Q') at (0.55,1.6);
        \node[right]() at (P') {$P'$};
        \node[right]() at (Q') {$Q'$};
        \draw[thick,{Circle[sep=-2pt]}-stealth] (P') -- (Q') node[midway,right]{$\vec{d}'$};
        \draw[closed,ultra thick] (z0) .. (z1) .. (z2) .. (z3) .. (z4);
    \end{scope}

    \draw[thick,-stealth] (P) -- (P') node[midway,above]{$\vec{u}(\vec{x})$};
    \draw[thick,-stealth] (Q) -- (Q') node[midway,above]{$\vec{u}(\vec{x}+\vec{d})$};

    \draw[thick,-stealth] (0,0) -- (P) node[midway,right]{$\vec{r}_0$};
    \draw[thick,-stealth] (0,0) -- (P') node[midway,below]{$\vec{r}$};
  
\end{tikzpicture}
 \caption{Legenda}
 \label{fig: campo_deslocamentos}
\end{figure}


Para determinar $\vec{u}$, considerem-se os pontos $P$ e $Q$ cujas posições inicias sejam, inicialmente, 
separadas pelo vetor $\vec{d}$. O vetor de deslocamento do ponto $P$ em direção à posição final, deformada,
$P'$ é $\vec{u}=\vec{r}-\vec{r}_0$ (Fig.~\ref{fig: campo_deslocamentos}). Assumindo-se a interpolação dada pela Eq.~\eqref{eq: posi_ANCF}, decorre que:
\begin{align}
    \vec{u} &= \vec{S}(\vec{\xi})\cdot\vec{q} \label{eq: campo_desl_ANCF} \\
    \delta'\dot{\vec{r}} &= \vec{S}(\vec{\xi})\cdot\delta'\dot{\vec{q}} \label{eq: var_velocidades_ANCF}
\end{align}

Substituindo as expressões anteriores na Eq.~\eqref{eq: eq_mov_2}:
\begin{equation}
     \delta'\dot{\vec{q}}^T \left\lbrace \vec{M}_i \ddot{\vec{q}} + 
    \int_{e^{(i)}}{ \vec{S}(\vec{\xi})^T \vec{L}^T \vec{D}\vec{L}\vec{S}(\vec{\xi})\vec{q}  dV}
    - \int_{\Gamma^{(i)}}{\vec{S}(\vec{\xi})^T \vec{t} d\Gamma} 
    - \int_{e^{(i)}}{\vec{S}(\vec{\xi})^T \vec{b} dV}\right\rbrace = 0  \label{eq: eq_mov_3}
\end{equation}

E, finalmente, como as velocidades virtuais $\delta'\dot{\vec{q}}^T$ são arbitrárias e condizentes com as 
condições de contorno, chega-se a:
\begin{equation}
     \vec{M}_i \ddot{\vec{q}} + 
    \int_{e^{(i)}}{ \vec{S}(\vec{\xi})^T \vec{L}^T \vec{D}\vec{L}\vec{S}(\vec{\xi})\vec{q}  dV}
    - \int_{\Gamma^{(i)}}{\vec{S}(\vec{\xi})^T \vec{t} d\Gamma} 
    - \int_{e^{(i)}}{\vec{S}(\vec{\xi})^T \vec{b} dV} = 0  \label{eq: eq_mov_4}
\end{equation}

Seja, agora, a matriz $\vec{S}_{,\vec{r}_0} = \vec{L}\vec{S}(\vec{\xi})$. As equações de movimento \eqref{eq: eq_mov_4} ficam:
\begin{equation}
     \vec{M}_i \ddot{\vec{q}} + 
    \left[ \int_{e^{(i)}}{ \vec{S}_{,\vec{r}_0}^T \vec{D} \vec{S}_{,\vec{r}_0}  dV}\right]\vec{q}
    - \int_{\Gamma^{(i)}}{\vec{S}(\vec{\xi})^T \vec{t} d\Gamma} 
    - \int_{e^{(i)}}{\vec{S}(\vec{\xi})^T \vec{b} dV} = 0  \label{eq: eq_mov_quase_final}
\end{equation}
com
\begin{equation}
    \vec{S}_{,\vec{r}_0} = \begin{bmatrix}
        \frac{\partial \vec{S}_x}{\partial r_{0x}} & \vec{0} & \vec{0} \\
        \vec{0} & \frac{\partial \vec{S}_y}{\partial r_{0y}} & \vec{0} \\
        \vec{0} & \vec{0} & \frac{\partial \vec{S}_z}{\partial r_{0z}} \\
        \frac{\partial \vec{S}_x}{\partial r_{0y}} & \frac{\partial \vec{S}_y}{\partial r_{0x}} & \vec{0} \\
        \frac{\partial \vec{S}_x}{\partial r_{0z}} & \vec{0} & \frac{\partial \vec{S}_z}{\partial r_{0x}} \\
        \vec{0} & \frac{\partial \vec{S}_y}{\partial r_{0z}} & \frac{\partial \vec{S}_z}{\partial r_{0y}}
    \end{bmatrix} \label{eq: matriz_B}
\end{equation}


A consideração de um elemento pré-deformado, ou seja, de formato inicialmente não prismático e não paralelo aos eixos cartesianos 
pode ser feita 
com o uso da matriz Jacobiana da transformação linear entre as coordenadas globais iniciais dos nós dos elementos e as coordenadas 
próprias do elemento.
\begin{equation} \vec{J}_0 = \begin{bmatrix} 
        \vec{S}_{x,\xi}\vec{q}_0 & \vec{S}_{x,\eta}\vec{q}_0 & \vec{S}_{x,\zeta}\vec{q}_0 \\
        \vec{S}_{y,\xi}\vec{q}_0 & \vec{S}_{y,\eta}\vec{q}_0 & \vec{S}_{y,\eta}\vec{q}_0 \\
        \vec{S}_{z,\xi}\vec{q}_0 & \vec{S}_{z,\eta}\vec{q}_0 & \vec{S}_{z,\eta}\vec{q}_0
\end{bmatrix} =  \begin{bmatrix} 
        \vec{S}_{x,\xi} & \vec{S}_{x,\eta} & \vec{S}_{x,\zeta} \\
        \vec{S}_{y,\xi} & \vec{S}_{y,\eta} & \vec{S}_{y,\eta} \\
        \vec{S}_{z,\xi} & \vec{S}_{z,\eta} & \vec{S}_{z,\eta}
\end{bmatrix} \begin{bmatrix}
    \vec{q}_0 & \vec{0} & \vec{0} \\
    \vec{0}& \vec{q}_0 & \vec{0} \\
    \vec{0} & \vec{0} & \vec{q}_0 \\
\end{bmatrix} = \vec{S}_{,\vec{\xi}}\vec{Q}_0\label{eq:jaco_0_3D} \end{equation}

A relação entre o volume inicial pré-deformado e o volume do elemento de referência, idealmente prismático, é o determinante $J_0=\det{\vec{J}_0}$,
de modo que a forma final das equações de movimento fica:
\begin{equation}
     \vec{M}_i \ddot{\vec{q}} J_0 + 
    \left[ \int_{e^{(i)}}{ \vec{S}_{,\vec{r}_0}^T \vec{D} \vec{S}_{,\vec{r}_0} J_0 dV}\right]\vec{q}
    - \int_{\Gamma^{(i)}}{\vec{S}(\vec{\xi})^T \vec{t} J_0 d\Gamma} 
    - \int_{e^{(i)}}{\vec{S}(\vec{\xi})^T \vec{b} J_0 dV} = 0  \label{eq: eq_mov_final}
\end{equation}