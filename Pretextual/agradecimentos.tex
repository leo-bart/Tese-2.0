\begin{agradecimentos}

\color{red}{Elemento opcional no qual o autor faz agradecimentos dirigidos àqueles que contribuíram de maneira relevante à elaboração do trabalho.}\\


Quando se tratar de dissertações e teses que receberam auxílio financeiro (integral ou parcial) de agências de fomento é obrigatória a referência ao apoio recebido . (OF PRPG 002/2019 – Orientação sobre dissertações e teses), usando as expressões, no idioma do trabalho, indicadas pelas Agências:\\

CAPES:\\
“O presente trabalho foi realizado com apoio da Coordenação de Aperfeiçoamento de Pessoal de Nível Superior – Brasil (CAPES) – Código de Financiamento 001.”\\
“This study was financed in part by the Coordenação de Aperfeiçoamento de Pessoal de Nível Superior – Brasil (CAPES) – Finance Code 001.”\\

FAPESP:\\
“O presente trabalho foi realizado com apoio da Fundação de Amparo à Pesquisa do Estado de São Paulo (FAPESP), processo aaaa/nnnnn-d.”\\
“This study was financed in part by the São Paulo Research Foundation (FAPESP), grant aaaa/nnnnn-d.”\\

CNPq e demais apoios, seguir modelo:\\
“O presente trabalho foi realizado com apoio da Conselho Nacional de Desenvolvimento Científico e Tecnológico (CNPq), processo no nnnnnn/aaaa-d.”\\
“This study was financed in part by The Brazilian National Council for Scientific and Technological Development (CNPq), grant nnnnnn/aaaa-d.”\\


Sugerimos que o nome dos órgãos de fomento não seja citado caso não tenha ocorrido financiamento ou uso de recursos de custeio.


\end{agradecimentos}
