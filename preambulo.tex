%ARQUIVO DE PREÂMBULO DA TESE - PACOTES E CONFIGURAÇÕES

\documentclass[12pt, oneside, a4paper, chapter=TITLE, english, brazil, sumario=tradicional, hyphens]{abntex2}
%	12pt	,  tamanho da fonte
%	openright,	capítulos começam em página impar
%	(insere página vazia caso necessário)
%	oneside,	para impressão somente frente
%	twoside, para impressão em frente e verso
%	a4paper,  tamanho do papel.
% -- ---opções da classe abntex2 ---------------
%	chapter=TITLE,	títulos de capítulos convertidos em
%				letras maiúsculas
%	section=TITLE,	títulos de seções convertidos em letras
%				maiúsculas
%	subsection=TITLE,	 títulos de subseções convertidos em
%				letras maiúsculas
%	subsubsection=TITLE,	títulos de subsubseções convertidos em
%					letras maiúsculas
%	PARA MODIFICAR TÍTULOS PARA  MINÚSCULAS, DEVE-SE SUBSTITUIR
%	A PALAVRA "TITLE" POR "title"
% ------ opções do pacote babel ---------------
%	english,	idioma adicional para hifenização
%	french,		idioma adicional para hifenização
%	spanish,	idioma adicional para hifenização
%	brazil,
%				O ÚLTIMO IDIOMA ESTABELECE A HIFENIZAÇÃO PRINCIPAL DO DOCUMENTO
%	sumario=tradicional,
%	hyphens
%----------------------------
%	as figuras e tabelas serão numeradas por capítulos, para numeração contínua, apagar as 	linhas abaixo
\counterwithin{figure}{chapter} 
\counterwithin{table}{chapter}
%---------------------------------------------------------------------------------------
%Lista de pacotes elementares
%---------------------------------------------------------------------------------------
\usepackage[alf, abnt-etal-cite=2, abnt-etal-list=0, abnt-etal-text=emph, abnt-and-type=e, abnt-last-names=bibtex, abnt-emphasize=bf]{abntex2cite}   %PACOTE ABNTEX PARA CITAÇÕES
%	alf,		citação por autor-data
% 	abnt-etal-cite=2,  abrevia as citações com mais de dois autores da forma "autor 1 et al."	
%	abnt-etal-list=0, 	abrevia lista de autores em trabalhos com mais de três autores
%				na bibliografia
%	abnt-etal-text=emph,		as palavras "et al." aparecerá na forma \emph{et al.}
%	abnt-and-type=e, 	usa o "e" para separar último autor
%	abnt-last-names=bibtex, 	os sobrenomes são tratados como no bibtex	
%	abnt-emphasize=bf,	títulos de revistas em negrito	






\usepackage[utf8]{inputenc} 	%caracteres especiais
\usepackage{amsmath, amssymb, amsfonts, amsthm} % fontes matemáticas para equações
\usepackage[left=3cm, right=2cm, top=3cm, bottom=2cm, footnotesep=0.5cm]{geometry}
		% dimensões das margens
\usepackage[T1]{fontenc}
\usepackage{graphicx} %inserir figuras e imagens
\usepackage{indentfirst} % recuo no primeiro parágrafo de casa seção
\usepackage{times} % mudar fonte do documento para estilo Times
\usepackage{pdfpages} % permite anexar documentos em formato PDF
\usepackage{unicamp-fem} % pacote de configuração FEM-UNICAMP
%---------------------------------------------------------------------------------------
%PACOTES ACRESCENTADOS PELO USUÁRIO
\usepackage{multirow} 
\usepackage{tikz,pgfplots}
    \usetikzlibrary{calc,
                    backgrounds,
                    decorations.pathreplacing,
                    patterns,
                    arrows.meta,
                    calc,
                    hobby}
    \pgfplotsset{compat=newest,
        /pgf/number format/.cd,
        use comma,
        1000 sep={}}
        \pgfdeclarelayer{background}
        \pgfdeclarelayer{foreground}
        \pgfsetlayers{background,main,foreground}
\usepackage{mathtools}
  \mathtoolsset{showonlyrefs}
\usepackage{booktabs}
\usepackage{siunitx}
    \sisetup{output-decimal-marker = {,},
            per-mode=symbol,
            }
%\usepackage[inline]{asymptote}
\usepackage{multirow,longtable}
\usepackage{nomencl}
    \makenomenclature
\usepackage{enumitem}
%---------------------------------------------------------------------------------------
%PACOTES PARA MANUTENÇÃO DO FORMATO (pode ser apagado)
%\usepackage{verbatim}
%\usepackage{xcolor}
\usepackage{lipsum}
%---------------------------------------------------------------------------------------
% Configurações de aparência do PDF final
%	Atraves das seguintes linhas são modificadas as cores nas
%	citações e links que são impressos no documento final
%- ALTERAÇÃO DA COR DAS REFERÊNCIAS CRUZADAS E CITAÇÕES
% -- alterando o aspecto da cor azul
%\definecolor{blue}{RGB}{41,5,195}

%-- informações do PDF
\makeatletter

\hypersetup{
	%As seguintes informações serão estabelecidas nas propriedades do
	%pdf final
     	%pagebackref=true,
	pdftitle={\@title},%Título que sera adotado nas propriedades do pdf
	pdfauthor={\@author},%Autors que aparecera nas propriedades do pdf
	pdfsubject={\imprimirpreambulo},%Descripção do pdf 
	pdfcreator={LaTeX with abnTeX2},
	pdfkeywords={abnt}{latex}{abntex}{abntex2}{trabalho acadêmico},
	colorlinks=true,       		% false: boxed links; true: colored links
    	linkcolor=black,          	% color of internal links
    	citecolor=blue,        		% color of links to bibliography
    	filecolor=black,      		% color of file links
	urlcolor=black,
	bookmarksdepth=4
}
\makeatother
%--------------------------------------------------------------------------------------
% CONFIGURAÇÕES DE PACOTES
%
\renewcommand{\ABNTEXchapterfont}{\fontfamily{ptm}\fontseries{b}\selectfont}
\renewcommand{\ABNTEXchapterfontsize}{\large}
\renewcommand{\ABNTEXsectionfont}{\fontfamily{ptm}\fontseries{b}\selectfont}
\renewcommand{\ABNTEXsectionfontsize}{\normalsize}
\renewcommand{\ABNTEXsubsectionfont}{\fontfamily{ptm}\selectfont}
\renewcommand{\ABNTEXsubsectionfontsize}{\normalsize}
%--------------------------------------------------------------------------------------

%--------------------------------------------------------------------------------------
% COMANDOS MEUS
%
\renewcommand{\vec}[1]{\boldsymbol{\mathrm{#1}}}
\renewcommand{\sin}{\mathrm{sen\,}}
\newcommand{\tens}[1]{\vec{#1}}
\newcommand{\tp}[1]{{#1}^{\mathrm{\scriptscriptstyle T}}}
\newcommand{\inv}[1]{{#1}^{-1}}
\providecommand{\sin}{} \renewcommand{\sin}{\hspace{2pt}\textrm{sen}}
\providecommand{\tan}{} \renewcommand{\tan}{\hspace{2pt}\textrm{tan}}


%--------------------------------------------------------------------------------------
% TIKZ
%
\tikzset{
cota/.style={stealth-stealth, thick}
                }
\tikzset{
suporte/.style = {shorten >= -4 pt, shorten <= 2pt}
         }

\tikzset{
    roda/.pic = {
        \draw[rounded corners = 0.8] (0,0) -- (0,1.8) -- (0.8,2.2) -- (0.8,2.3) --
            (0.9,2.3) -- (0.9,0) -- 
            (0.9,-2.3) -- (0.8,-2.3) -- (0.8,-2.2) --
            (0,-1.8) -- cycle;
            \draw[thin] (0.8,2.2) -- (0.8,-2.2);
    }    
}
\tikzset{
    rodeiro/.pic = {
        \pic[shift={(-4,0)}]{roda};
        \pic[shift={(4,0)},rotate=180]{roda};
        \draw (-3.1,0.3) rectangle (3.1,-0.3);
        \draw[shorten <= -2mm,shorten >= -2mm,dash dot] (-4,0) -- (4,0);
    }
}

\tikzset{
    trilho/.pic = {
        \draw[thick] (1.233,5.682) arc (-75.964:1.432:5/16) -- (1.449,6.756) arc (1.432:68.597:9/16)
            arc (68.597:87.134:1.25) arc (87.134:90:14);
        \begin{scope}[xscale=-1,yscale=1]
            \draw[thick] (1.233,5.682) arc (-75.964:1.432:5/16) -- (1.449,6.756) arc (1.432:68.597:9/16)
            arc (68.597:87.134:1.25) arc (87.134:90:14);
        \end{scope}

    }
}

\tikzset{
    trilhoTotal/.pic = {
        \draw[] (0,0) -- (3-1/16,0) arc (270:360:1/16) -- (3,7/16) arc (0:75.964:1/8) -- 
            (1.042,0.927) arc (255.964:186.623:3/4) arc (186.623:180:20) arc (180:171.246:8)
            arc (171.246:139.418:3/4) arc (139.418:104.036:5/16) -- (1.233,5.682)
            arc (-75.964:1.432:5/16) -- (1.449,6.756) arc (1.432:68.597:9/16) arc (68.597:87.134:1.25) 
            arc (87.134:90:14);
            \draw[xscale=-1] (0,0) -- (3-1/16,0) arc (270:360:1/16) -- (3,7/16) arc (0:75.964:1/8) -- 
            (1.042,0.927) arc (255.964:186.623:3/4) arc (186.623:180:20) arc (180:171.246:8)
            arc (171.246:139.418:3/4) arc (139.418:104.036:5/16) -- (1.233,5.682)
            arc (-75.964:1.432:5/16) -- (1.449,6.756) node (A) {} arc (1.432:68.597:9/16) arc (68.597:87.134:1.25) 
            arc (87.134:90:14);
    }
}

\newcommand{\trilhoperfil}[1]{
    \colorlet{pen}{#1}
    \draw[pen,thick] (1.233,5.682) arc (-75.964:1.432:5/16) -- (1.449,6.756) arc (1.432:68.597:9/16)
        arc (68.597:87.134:1.25) arc (87.134:90:14);

}
\newcommand{\meiotrilho}{
    \draw[thick, dashed] (0,0) -- (3-1/16,0) arc (270:360:1/16) -- (3,7/16) arc (0:75.964:1/8) -- 
        (1.042,0.927) arc (255.964:186.623:3/4) arc (186.623:180:20) arc (180:171.246:8)
        arc (171.246:139.418:3/4) arc (139.418:104.036:5/16) -- (1.233,5.682)
        arc (-75.964:1.432:5/16) -- (1.449,6.756) arc (1.432:68.597:9/16) arc (68.597:87.134:1.25) 
        arc (87.134:90:14);
    \trilhoperfil{red}
}